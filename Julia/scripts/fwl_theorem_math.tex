\documentclass{article}
\usepackage{amsmath}
\usepackage{amssymb}
\usepackage{amsthm}

\title{Math Section: Frisch-Waugh-Lovell Theorem}
\author{High Dimensional Linear Models Assignment}
\date{}

\newtheorem{theorem}{Theorem}
\newtheorem{proof}{Proof}

\begin{document}

\maketitle

\section{Math (3 points)}

\subsection{Linear Regression Model}

Consider the linear regression model:
\begin{equation}
y = X_1\beta_1 + X_2\beta_2 + u
\end{equation}

where:
\begin{itemize}
    \item $y$ is an $n \times 1$ vector of outcomes
    \item $X_1$ is an $n \times k_1$ matrix of regressors of interest  
    \item $X_2$ is an $n \times k_2$ matrix of control variables
    \item $u$ is an $n \times 1$ vector of errors
\end{itemize}

\subsection{Frisch-Waugh-Lovell Theorem}

\begin{theorem}[Frisch-Waugh-Lovell Theorem]
The OLS estimate of $\beta_1$ in the regression of $y$ on $[X_1 \quad X_2]$ is equal to the OLS estimate obtained from the following two-step procedure:

\begin{enumerate}
    \item Regress $y$ on $X_2$ and obtain the residuals $\tilde{y} = M_{X_2}y$, where $M_{X_2} = I - X_2(X_2'X_2)^{-1}X_2'$
    \item Regress $X_1$ on $X_2$ and obtain the residuals $\tilde{X_1} = M_{X_2}X_1$  
    \item Regress $\tilde{y}$ on $\tilde{X_1}$ and show that the resulting coefficient vector is equal to $\hat{\beta_1}$ from the full regression
\end{enumerate}

Formally, we need to show that:
\begin{equation}
\hat{\beta_1} = (\tilde{X_1}'\tilde{X_1})^{-1}\tilde{X_1}'\tilde{y}
\end{equation}
\end{theorem}

\begin{proof}
We will prove the FWL theorem by showing that the two-step partialling-out procedure yields the same coefficient estimate as the full regression.

\textbf{Step 1: Set up the full regression}

The full regression in matrix form is:
\begin{equation}
y = [X_1 \quad X_2]\begin{bmatrix} \beta_1 \\ \beta_2 \end{bmatrix} + u = X\beta + u
\end{equation}

where $X = [X_1 \quad X_2]$ and $\beta = [\beta_1' \quad \beta_2']'$.

The OLS estimator is:
\begin{equation}
\hat{\beta} = (X'X)^{-1}X'y
\end{equation}

\textbf{Step 2: Partition the matrices}

We can partition the matrices as:
\begin{equation}
X'X = \begin{bmatrix} X_1'X_1 & X_1'X_2 \\ X_2'X_1 & X_2'X_2 \end{bmatrix}
\end{equation}

\begin{equation}
X'y = \begin{bmatrix} X_1'y \\ X_2'y \end{bmatrix}
\end{equation}

\textbf{Step 3: Apply the partitioned matrix inverse formula}

Using the partitioned matrix inverse formula:
\begin{equation}
(X'X)^{-1} = \begin{bmatrix} (X_1'M_{X_2}X_1)^{-1} & -(X_1'M_{X_2}X_1)^{-1}X_1'X_2(X_2'X_2)^{-1} \\ \text{...} & \text{...} \end{bmatrix}
\end{equation}

where $M_{X_2} = I - X_2(X_2'X_2)^{-1}X_2'$ is the annihilator matrix.

\textbf{Step 4: Extract the coefficient of interest}

From the partitioned inverse, the coefficient estimate for $\beta_1$ is:
\begin{equation}
\hat{\beta_1} = (X_1'M_{X_2}X_1)^{-1}X_1'M_{X_2}y
\end{equation}

\textbf{Step 5: Show equivalence to the two-step procedure}

Now we show this equals the two-step procedure:

From Step 1 of the procedure: $\tilde{y} = M_{X_2}y$

From Step 2 of the procedure: $\tilde{X_1} = M_{X_2}X_1$

From Step 3 of the procedure:
\begin{equation}
\hat{\beta_1}^{FWL} = (\tilde{X_1}'\tilde{X_1})^{-1}\tilde{X_1}'\tilde{y}
\end{equation}

Substituting the definitions:
\begin{align}
\hat{\beta_1}^{FWL} &= ((M_{X_2}X_1)'(M_{X_2}X_1))^{-1}(M_{X_2}X_1)'(M_{X_2}y) \\
&= (X_1'M_{X_2}'M_{X_2}X_1)^{-1}X_1'M_{X_2}'M_{X_2}y
\end{align}

Since $M_{X_2}$ is symmetric and idempotent ($M_{X_2}' = M_{X_2}$ and $M_{X_2}M_{X_2} = M_{X_2}$):
\begin{align}
\hat{\beta_1}^{FWL} &= (X_1'M_{X_2}X_1)^{-1}X_1'M_{X_2}y \\
&= \hat{\beta_1}
\end{align}

Therefore, we have shown that:
\begin{equation}
\hat{\beta_1} = (\tilde{X_1}'\tilde{X_1})^{-1}\tilde{X_1}'\tilde{y}
\end{equation}

This completes the proof of the Frisch-Waugh-Lovell theorem.
\end{proof}

\subsection{Economic Interpretation}

The FWL theorem demonstrates that we can obtain the same coefficient estimate for $\beta_1$ by:
\begin{enumerate}
    \item "Partialling out" the effect of $X_2$ from both $y$ and $X_1$
    \item Running a simple regression on the residuals
\end{enumerate}

This is particularly useful for:
\begin{itemize}
    \item Understanding the role of control variables
    \item Computational efficiency in high-dimensional settings
    \item Theoretical analysis of partial correlation
\end{itemize}

\end{document}